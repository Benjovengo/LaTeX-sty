\documentclass{book}

\usepackage[red]{benjovengo}

\usepackage{lipsum}

\makeindex

\title{\bfseries\textcolor{textgray}{Sistemas de Controle e Automação} \textcolor{gray}{$|$} \textcolor{MyDarkYellow}{Puc Campinas}}
\author{\bfseries Prof. Dr. Fábio Pereira Benjovengo}

\begin{document}

\frontmatter

\maketitle % Cover
\tableofcontents % Print the table of contents
\listoffigures % Print a list of figures

\mainmatter
\chapter{Título 1}
\section{Introduction}
In this document a new package is tested. This package allows special numbered environments

\begin{example}
This text is inside a special environment, some boldface text is printed
at the beginning and a new indentation is set.
\end{example}

Also, there's a special command for \important{important!words} that will be printed in a special \important{colour} depending on the parameter used in the \important{package} importation statement. Because it's \important{important}.

\begin{warning}
How to get some \textsc{Mathematica} code (shown below) in this box...\\
\lipsum[1][1]
\end{warning}

\subsection{Introdução mais específica}

Teste de subseção.

\subsubsection{Último nível}

Deve ter uma pequena diferença. Agora sem o bold.

\printindex

\end{document}
